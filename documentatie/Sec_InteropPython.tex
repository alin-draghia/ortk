\section{Interoperabilitatea cu Python}
Interoperabilitatea cu limbajul Python se poate realiza folosind una dinte aceste tehnologii:
\begin{itemize}
	\item Python C Api\cite{pythonCAPI}
	\item SWIG\cite{swig}
	\item pyrex\cite{pyrex}
	\item ctypes\cite{ctypes}
	\item SIP\cite{sip}
	\item Boost.Python\cite{boostpython}
\end{itemize}

Am ales sa rezolv problema de interoperabilitate cu limbajul Python prin intermediul bibliotecii Boost.Python.
Particularitățile acestei biblioteci vor fi discutate în capitolul "Tehnologii folosite".

Pentru ca o clasa abstracta declarata în C++ sa poate fi implementata in Python, iar aceasta implementare sa fie disponibila atât în Python cat și în C++ este nevoie de o clasa ajutătoare care sa acționeze ca o legătura intre cele doua limbaje. 
Aceasta clase se implementează prin moștenire din clasa pe care dorim sa o expunem și clasa "wrapper" din Boost.Python.
O astfel de implementare poate fi observata in codul urmator:
\begin{lstlisting}[language=C++]
namespace object_recognition_toolkit
{
	namespace pyramid
	{
		namespace bp = boost::python;

		struct PyramidBuilder_Wrapper : public PyramidBuilder
			, public bp::wrapper < PyramidBuilder > {
			
			virtual ~PyramidBuilder_Wrapper() {}

			Pyramid Build(core::Matrix const& image) const
			{ return this->get_override("Build")(image); }

			boost::shared_ptr<PyramidBuilder> Clone() const
			{ return this->get_override("Clone")(); }
		};
	}
}
\end{lstlisting}

Pe urma, acesta clasa trebuie înregistrată pentru a fi vizibila în Python.
Înregistrarea se face cu Boost.Python prin intermediul clasei \verb!class_! după cum se observa mai jos:

\begin{lstlisting}[language=C++]
void py_regiser_pyramid()
{
using namespace boost::python;
using namespace object_recognition_toolkit::core;
using namespace object_recognition_toolkit::pyramid;
using object_recognition_toolkit::python_ext::serialize_pickle;

class_<PyramidLevel>("PyramidLevel", 
		init<const Matrix&, double>((arg("image"),arg("scale"))))
	.def("GetScale", &PyramidLevel::GetScale)
	.def("GetImage", &PyramidLevel::GetImage, 
		return_value_policy<copy_const_reference>())
	.def("Transform", &PyramidLevel::Transform<int>, arg("box"))
	.def("Invert", &PyramidLevel::Invert<int>, arg("box"))
	;

class_<Pyramid>("Pyramid", init<>())
	.def("AddLevel", &Pyramid::AddLevel, arg("level"))
	.def("Clear", &Pyramid::Clear)
	.def("GetNumLevels", &Pyramid::GetNumLevels)
	.def("GetLevel", &Pyramid::GetLevel, arg("index"), 
		return_value_policy<copy_const_reference>())
	;

class_<PyramidBuilder_Wrapper, boost::noncopyable>("PyramidBuilder")
	.def("Clone", pure_virtual(&PyramidBuilder::Clone))
	.def("Build", pure_virtual(&PyramidBuilder::Build), arg("image"))
	.enable_pickling()
	;
register_ptr_to_python<boost::shared_ptr<PyramidBuilder>>();

class_<FloatPyramidBuilder, bases<PyramidBuilder>>("FloatPyramidBuilder", init<>())
	.def(init<double, Size, Size>(args("scale_factor", "min_size", "max_size")))
	.def_readwrite("scale_factor", &FloatPyramidBuilder::scaleFactor_)
	.def_readwrite("max_size", &FloatPyramidBuilder::maxSize_)
	.def_readwrite("min_size", &FloatPyramidBuilder::minSize_)
	.def_pickle(serialize_pickle<FloatPyramidBuilder>())
	;
}
\end{lstlisting}

După care, implementarea aceste clase poate fi observata in codul urmator:
\begin{lstlisting}[language=Python,frame=single]
import object_recognition_toolkit as ort

class MyPyramidBuilder(ort.PyramidBuilder):

    def __init__(self, num):
        ort.PyramidBuilder.__init__(self)
        self.num = num
        
    def Build(self, image):
        pyr = ort.Pyramid()
        for si in range(self.num):
        	scale = 1.0//(2.0**float(si))
        	level = ort.PyramidLevel(image, scale)
            pyr.AddLevel(levele)
        return pyr
    
    #implement pickle support
    def __reduce__(self):
        initargs = (self.num, )
        # this is not necesary, becouse num is an init arg, 
        # but for the sake of example :)
        state = {'num': self.num} 
        return (MyPyramidBuilder, initargs, state, )
    
    def __setstate__(self, state):
        self.num = state['num']
\end{lstlisting}