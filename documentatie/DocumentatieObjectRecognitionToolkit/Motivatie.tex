\section{Motivație}

Recunoașterea obiectelor în cadrul inteligentei artificiale reprezinta localizarea și identificarea obiectelor într-o imagine sau o secventa video.
Oamenii pot recunoaște o mulțime de obiecte într-o imagine cu fără sa depună prea mult efort, chiar dacă în aceste imagini obiectele prezintă variații de perspectiva, de dimensiune, sunt translatate, rotite sau chiar obstrucționate.
Aceasta problema nu poate fii nici pe departe considerata rezolvata, de-a lungul timpului un număr mare de algoritmi au fost propuși.

Dezvoltarea rapida a sistemelor de calcul a permis utilizarea acestor algoritmi în tot mai multe aplicații, pornind de la aplicații industriale pana la cele medicale sau chiar în pagini pe internet.

Dezvoltarea rapida a sistemelor de calcul a permis utilizarea acestor algoritmi în tot mai multe aplicații:
\begin{itemize}
	\item Industriale: recunoașterea și verificarea cip-urilor pe o placa electronica
	\item Securitate: recunoașterea unui intrus folosind o camera de supraveghere
	\item Medicale: recunoașterea diferitelor tumori într-o imagine de tomografie
	\item Camere Fote: focalizare automata pe fete
	\item Internet: căutare google după imagini
\end{itemize}