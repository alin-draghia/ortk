\section{Tehnologii folosite}

\subsection{Limbajul C++}

Limbajul C++ este un limbaj de programare general care și este compilat în cod-mașina.
Este un limbaj multi-paradigma, cu verificare statica a tipurilor. 
Suporta programarea procedurala, orientata pe obiecte și generica.
Limbajul oferă facilitați de manipulare a memoriei la nivel scăzut.
Fiind proiectat inițial ca un limbaj pentru programarea de sisteme(sisteme integrate, kernel sisteme de operare), performața și eficienta sunt trăsături principale.

Dat fiind faptul ca este și compatibil cu limbajul C, utilizatorii C++ au la dispoziție o gama larga de librarii software din cele mai diverse ramuri de aplicații de care se pot folosii.

\subsection{Limbajul Python}



\subsection{Limbajele C++ și Python}
Pentru realizarea lucrării am ales sa folosesc C++ și Python din mai multe motive:

C++ și Python sunt doua limbaje de programare atât de diferite încât putem spune ca se afla în capete diferite ale axei limbajelor de programare.
\begin{itemize}
	\item C++ este compilat în cod-mașina, Python este interpretat
	\item Python are sistemul de tipuri dinamic și este recunoscut pentru flexibilitate
	\item C++ are sistemul de tipuri static și este recunoscut pentru eficienta
	\item Python eliberează automat memoria
\end{itemize}

Pentru multi programatori, aceste diferențe înseamna ca cele doua limbaje se complementează perfect.

\subsection{Librăria OpenCV}

Librăria OpenCV este cea mai populara librărie de procesare de imagini


\subsection{Librăria Boost}

\subsection{Librăria scikit-learn}

\subsection{Librăria Qt}




