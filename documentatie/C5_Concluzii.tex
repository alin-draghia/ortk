\chapter{Concluzii}

În aceasta lucrare am prezentat în ce consta dezvoltarea unui sistem de recunoaștere automata a obiectelor în imagini. Au fost prezentate în detaliu componentele din care este alcătuit un astfel de sistem. De asemenea am prezentat un cadru de lucru pentru dezvoltarea de algoritmi și aplicații de recunoaștere a obiectelor.

Algoritmii și aplicațiile implementate folosind acest cadru de lucru pot fii scrise în C++ și Python.
Acest lucru permite dezvoltarea rapida a aplicațiilor, iar atunci când este nevoie de performanta se pot rescrie parți din algoritm sau aplicație în C++.
Totuși exista foarte multi algoritmi implementați în limbaje precum Java și Matlab pe care nu ii putem folosi în mod direct.
O direcție de dezvoltare ar ca cadrul de lucru sa permită dezvoltarea și în alte limbaje de programare. Acest lucru s-ar putea realiza folosind tehnici din programarea distribuita: RMI, CORBA sau COM.

Folosind acest cadru am implementat un algoritm de recunoaștere a persoanelor în imagini, o aplicație de antrenare și una care sa-l folosească.
Astfel cadrul de lucru și-a atins scopul de a servi în dezvoltarea de algoritmi și aplicații.
Totuși algoritmul nu poate recunoaște decât obiecte rigide.
O direcție de dezvoltare ar fi: identificarea și implementarea altor algoritmi sau tehnici de recunoaștere a obiectelor.

O alta direcție de dezvoltare a fi implementarea unei aplicații care sa permită oricui sa antreneze algoritmi fără a avea cunoștințe în domeniu. Aceasta aplicație ar putea avea o interfața web și ar putea funcționa ca un software as a service.




