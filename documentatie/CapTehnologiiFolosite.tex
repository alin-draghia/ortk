\chapter{Tehnologii folosite}

\section{Limbajul C++}
[Am rezervat o pagina]
\pagebreak


%Limbajul C++ este un limbaj de programare general care și este compilat în cod-mașina.
%Este un limbaj multi-paradigma, cu verificare statica a tipurilor. 
%Suporta programarea procedurala, orientata pe obiecte și generica.
%Limbajul oferă facilitați de manipulare a memoriei la nivel scăzut.
%find proiectat inițial ca un limbaj pentru programarea de sisteme(sisteme integrate, kernel sisteme de operare), performața și eficienta sunt trăsături principale.

%Dat find faptul ca este și compatibil cu limbajul C, utilizatorii C++ au la dispoziție o gama larga de librarii software din cele mai diverse ramuri de aplicații de care se pot folosii.

\section{Limbajul Python}
[Am rezervat o pagina]
\pagebreak


%\subsubsection{Limbajele C++ și Python}

%Am ales sa combin aceste doua limbaje de programare care, chiar dacă sunt foarte diferite din multe puncte de vedere, se complementează intre ele foarte bine.

%Fiecare limbaj are avantajele și dezavantajele lui.


%Pentru realizarea lucrării am ales sa folosesc C++ și Python din mai multe motive:

%C++ și Python sunt doua limbaje de programare atât de diferite încât putem spune ca se afla în capete diferite ale axei limbajelor de programare.
%\begin{itemize}
%	\item C++ este compilat în cod-mașina, Python este interpretat
%	\item Python are sistemul de tipuri dinamic și este recunoscut pentru flexibilitate
%	\item C++ are sistemul de tipuri static și este recunoscut pentru eficienta
%	\item Python eliberează automat memoria
%\end{itemize}

%Pentru multi programatori, aceste diferențe înseamna ca cele doua limbaje se complementează perfect.

%\subsubsection{Librăria OpenCV}

%Librăria OpenCV este cea mai populara librărie de procesare de imagini


\section{Biblioteca Boost}
[Am rezervat o pagina]
\pagebreak


\section{Biblioteca scikit-learn}
[Am rezervat o pagina]
\pagebreak

\section{Biblioteca numpy}
[Am rezervat o pagina]
\pagebreak

\section{Biblioteca matplotlib}
[Am rezervat o pagina]
\pagebreak

\section{Biblioteca PySide}
[Am rezervat o pagina]
\pagebreak