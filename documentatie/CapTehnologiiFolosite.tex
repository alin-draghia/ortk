\chapter{Tehnologii folosite}

În acest capitol voi enumera tehnologiile folosite și voi oferi cate o scurta descriere pentru fiecare.

\section{Limbajul C++}

C++ este un limbaj de programare general, multi-paradigma, compilat cu verificare statica a tipurilor.
Suporta programarea: procedurala, orientata pe obiecte și generica.
Limbajul oferă acces la facilitați de manipulare a memoriei pana la cel mai scăzut nivel.

Câteva dintre trăsăturile limbajului care îl fac potrivit pentru aceasta lucrare sunt:
\begin{itemize}
	\item Performanta. Multi dintre algoritmii ce vor fi descriși au de procesat un volum foarte mare de date. Este important ca execuția sa fie cat mai rapida.
	\item Portabilitate. Ne oferă posibilitatea de a porta codul către alte platforme.
	\item Biblioteci. Sunt scrise multe biblioteci pentru acest limbaj. În plus faptul ca este compatibil cu C ne pune practic la dispoziție un întreg arsenal de biblioteci.
	\item Modern. În ultimii ani limbajul a trecut printr-o întreaga serie de schimbări care l-au transformat într-un limbaj mult mai ușor de folosit.
\end{itemize}


\section{Limbajul Python}

Limbajul Python este un limbaj interpretat, dinamic, foarte ușor de folosit.
Ca și în cazul C++, Python este un limbaj multi-paradigma și suporta programarea orientata pe obiecte, procedurala și generica.
Este portabil și vine însoțit de o librărie standard vasta, pornind de la facilitați de lucru cu fișiere xml pana la cele de programare distribuita.
Vine integrat cu un manager de pachete: bibliotecile pot fi instalate mult mai ușor.
Are toate beneficiile unui limbaj modern: management automat de memorie, serializare automata a obiectelor și reflexia tipurilor.

Python este un limbaj foarte popular în domeniul învățării automate, de aceea sunt disponibile foarte multe biblioteci care susțin acest domeniu.
Este și un limbaj care se scalează de la mici aplicații demonstrative pana la aplicații distribuite cu sute de noduri.


%\subsubsection{Limbajele C++ și Python}

%Am ales sa combin aceste doua limbaje de programare care, chiar dacă sunt foarte diferite din multe puncte de vedere, se complementează intre ele foarte bine.

%Fiecare limbaj are avantajele și dezavantajele lui.


%Pentru realizarea lucrării am ales sa folosesc C++ și Python din mai multe motive:

%C++ și Python sunt doua limbaje de programare atât de diferite încât putem spune ca se afla în capete diferite ale axei limbajelor de programare.
%\begin{itemize}
%	\item C++ este compilat în cod-mașina, Python este interpretat
%	\item Python are sistemul de tipuri dinamic și este recunoscut pentru flexibilitate
%	\item C++ are sistemul de tipuri static și este recunoscut pentru eficienta
%	\item Python eliberează automat memoria
%\end{itemize}

%Pentru multi programatori, aceste diferențe înseamna ca cele doua limbaje se complementează perfect.

%\subsubsection{Librăria OpenCV}

%Librăria OpenCV este cea mai populara librărie de procesare de imagini


\section{Biblioteca Boost}

Este una dintre cele mai mari și de calitate biblioteci C++.
Din biblioteca Boost am folosit Boost.Serialization si Boost.Python, cu ajutorul cărora am implementat serializarea obiectelor și interoperabilitatea intre C++ și Python pentru sistemul dezvoltat în aceasta lucrare.

\section{Biblioteca OpenCV}

Este cea mai populara biblioteca de procesare a imaginilor. Aceasta implementează unul dintre cei mai folosiți algoritmi de recunoaștere a obiectelor în imagine și furnizează un model gata antrenat pentru recunoașterea fetelor. Pe lângă procesare de imagini, sunt oferite și funcționalități de învățare automata.

\section{Biblioteca scikit-learn}

Este cea mai cunoscuta biblioteca Python de învățare automata. Sunt implementați zeci de algoritmi de învățare, de selecție de trăsături, de validare și de reducere a dimensionalitatii.


\section{Biblioteca numpy}

Biblioteca numpy a transformat Python-ul într-un adevărat rival al MATLAB.
Aceasta biblioteca face lucru cu vectori și matrici in Python foarte simplu și eficient.


\section{Biblioteca matplotlib}

Matplotlib este o biblioteca Python de vizualizare a datelor. Se pot afișa imagini, trasa grafice sau vizualiza grafuri.

\section{Biblioteca PySide}

Biblioteca PySide este interfața de programare în Python a bibliotecii Qt, una dintre cele mai populare biblioteci de dezvoltare a interfețelor de utilizator portabile. 
Deși este cunoscuta pentru dezvoltarea de interfețe grafice, Qt este de fapt un cadru de lucru complet de dezvoltare de aplicații cu facilitați de lucru cu baze de date, fire de execuție, comunicare în rețea sau grafica 3d.
