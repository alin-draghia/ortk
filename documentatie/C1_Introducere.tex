\chapter{Introducere}

Oamenii pot recunoaște o mulțime de obiecte într-o imagine fără sa depună prea mult efort, chiar dacă în aceste imagini obiectele prezintă variații de perspectiva, de dimensiune, sunt translatate, rotite sau chiar obstrucționate.

Recunoașterea obiectelor este studiata în cadrul viziunii artificiale\footnote{Computer Vision}, o ramura a inteligentei artificiale.

În cadrul inteligentei artificiale recunoașterea obiectelor reprezinta capacitatea unui sistem software de localizare și identificare a obiectelor într-o imagine sau o secventa video.

Din punct de vedere practic, aceasta lucrare își propune dezvoltarea unei biblioteci software și aplicații demonstrative, cu ajutorul cărora sa se poată dezvolta aplicații de recunoașterea obiectelor în imagini.

Din punct de vedere teoretic sunt descrise componentele și structura unui astfel algoritm, precum și cel folosit pentru al antrena.


\section{Motivație}


Deși, de-a lungul timpului, au fost dezvoltați un număr mare de algoritmi care adresează aceasta problema, ea nu poate fii considerata încă rezolvata.



Aceasta problema nu poate fii nici pe departe considerata rezolvata, de-a lungul timpului un număr mare de algoritmi au fost propuși.

Dezvoltarea rapida a sistemelor de calcul a permis utilizarea acestor algoritmi în tot mai multe aplicații, pornind de la aplicații industriale pana la cele medicale sau chiar în pagini pe internet.

Dezvoltarea rapida a sistemelor de calcul a permis utilizarea acestor algoritmi în tot mai multe aplicații:
\begin{itemize}
	\item Industriale: recunoașterea și verificarea cip-urilor pe o placa electronica
	\item Securitate: recunoașterea unui intrus folosind o camera de supraveghere
	\item Medicale: recunoașterea diferitelor tumori într-o imagine de tomografie
	\item Camere Fote: focalizare automata pe fete
	\item Internet: căutare google după imagini
\end{itemize}



\section{Enunțul problemei}
Se scrie o librărie software cu ajutorul carea sa se antreneze și sa se folosească algoritmi de recunoaștere a obiectelor în imagini.

Algoritmul va învață sa recunoască obiecte folosindu-se de un set de exemple pozitive cat și negative.

Se scrie o aplicate care antrenează un algoritm de recunoaștere și îl salvează modelul învățat pe disc și una care încarcă modelul și îl aplica pe o imagine data.


\section{Tehnologii folosite}

\subsubsection{Limbajul C++}

Limbajul C++ este un limbaj de programare general care și este compilat în cod-mașina.
Este un limbaj multi-paradigma, cu verificare statica a tipurilor. 
Suporta programarea procedurala, orientata pe obiecte și generica.
Limbajul oferă facilitați de manipulare a memoriei la nivel scăzut.
Fiind proiectat inițial ca un limbaj pentru programarea de sisteme(sisteme integrate, kernel sisteme de operare), performața și eficienta sunt trăsături principale.

Dat fiind faptul ca este și compatibil cu limbajul C, utilizatorii C++ au la dispoziție o gama larga de librarii software din cele mai diverse ramuri de aplicații de care se pot folosii.

\subsubsection{Limbajul Python}



\subsubsection{Limbajele C++ și Python}
Pentru realizarea lucrării am ales sa folosesc C++ și Python din mai multe motive:

C++ și Python sunt doua limbaje de programare atât de diferite încât putem spune ca se afla în capete diferite ale axei limbajelor de programare.
\begin{itemize}
	\item C++ este compilat în cod-mașina, Python este interpretat
	\item Python are sistemul de tipuri dinamic și este recunoscut pentru flexibilitate
	\item C++ are sistemul de tipuri static și este recunoscut pentru eficienta
	\item Python eliberează automat memoria
\end{itemize}

Pentru multi programatori, aceste diferențe înseamna ca cele doua limbaje se complementează perfect.

\subsubsection{Librăria OpenCV}

Librăria OpenCV este cea mai populara librărie de procesare de imagini


\subsubsection{Librăria Boost}

\subsubsection{Librăria scikit-learn}

\subsubsection{Librăria Qt}



\section{Structura Lucrării}
