\chapter{Introducere}

Aceasta lucrare își propune sa prezinte, din punct de vedere atât teoretic cat și practic în ce consta dezvoltarea unui algoritm de recunoaștere a obiectelor în imagini, folosind tehnici de procesare a imaginilor și învățare automata.



\section{Motivație}

Oamenii pot recunoaște o mulțime de obiecte într-o imagine fără sa depună prea mult efort, chiar dacă în aceste imagini obiectele prezintă variații de perspectiva, de dimensiune, sunt translatate, rotite sau chiar obstrucționate.

Recunoașterea obiectelor este una dintre principalele aplicații ale viziunii artificiale\footnote{Computer vision} și procesarea de imagini. 

Viziunea artificiala reprezinta procesul invers al celui de formare a imagini și se ocupa cu recuperarea de informații din imagini cu ajutorul metodelor matematice, geometrice, statistice și a teoriei învățării automate\footnote{Machine learning}.

%Recunoașterea automata a obiectelor se refera la capacitatea unui sistem software de localizare și identificare a obiectelor într-o imagine sau o secventa video.

%Din punct de vedere practic, aceasta lucrare își propune dezvoltarea unei biblioteci software și aplicații demonstrative, cu ajutorul cărora sa se poată dezvolta aplicații de recunoașterea obiectelor în imagini.

%Din punct de vedere teoretic sunt descrise componentele și structura unui astfel algoritm, precum și cel folosit pentru al antrena.



Viziunea artificiala și învățarea automata sunt doua domenii aflate în plina dezvoltare și sunt de mare interes atât în cadrul academic ca și în industria software.

Recunoașterea obiectelor este una dintre aplicațiile fundamentale ale viziunii artificiale. Deși de-a lungul timpului au fost dezvoltați multi algoritmi, recunoașterea obiectelor este încă departe de a putea fii considerata o problema rezolvata.


Dezvoltarea rapida a sistemelor de calcul a permis utilizarea acestor algoritmi în tot mai multe aplicații. Câteva dintre aplicațiile recunoașterii de obiecte sunt:
\begin{itemize}
	\item Industriale: recunoașterea și verificarea cip-urilor pe o placa electronica, numărarea de obiecte pe o banda rulanta
	\item Securitate: recunoașterea unui intrus folosind o camera de supraveghere
	\item Medicale: recunoașterea diferitelor tumori într-o imagine de tomografie
	\item Fotografie: focalizare automata pe fete
	\item Internet: căutare google după imagini, marcarea automata a fetelor într-o poza de pe facebook
\end{itemize}

%Interesul 

%Pana acum am folosit și am studiat o mulțime de algoritmi de recunoaștere a obiectelor, dar pentru a aprofunda înțelegerea acestor algoritmi cel mai bine este sa fie scris măcar unul de la un capăt la altul.

%Cei mai multi algoritmi de recunoaștere a obiectelor au o structura comuna formata din următoarele componente: scanarea imaginii, extragerea de trasaturi, classificare si procesarea rezultatelor.

%Multi algoritmi sunt scrisi intr-un mod foarte rigid si componentele lor nu pot fii refolosite. Sunt optimizati pana la punctul in care codul sursa nu mai poate fii inteles cu usurinta.

%In cadrul companiei la care lucrez, Dynamic Ventures, am folosi

%Desi, am folosit si m-am documentat 

%Am ales sa dezvolt o astfel de librarie, chiar daca exista si altele, din doua motive.





%Aceasta problema nu poate fii nici pe departe considerata rezolvata, de-a lungul timpului un număr mare de algoritmi au fost propuși.

%Mult timp recunoașterea automata a obiectelor în imagini a fost considera impracticabila datorita complexității de timp și spațiu a acestor algoritmi.






\section{Enunțul problemei}
Se scrie o biblioteca software cu ajutorul căreia sa se dezvolte algoritmi și aplicații de recunoaștere a obiectelor.

Aceasta biblioteca va fii scrisa într-un mod hibrid, cu componente scose atât în C++ cat și în Python.

Toate componentele bibliotecii vor suporta serializare, pentru a putea fii salvate pe disc, baze de data sau trimise prin rețea în cazul unor programe distribuite.

Algoritmul va învață sa recunoască obiecte folosindu-se de un set de imagini cu exemple pozitive adnotate și exemple negative, imagini care nu conțin obiectul pe care dorim sa-l învățam.

Se scrie o aplicate care antrenează un algoritm de recunoaștere și îl salvează modelul învățat pe disc și una care încarcă modelul și îl aplica pe o imagine data.


\section{Tehnologii folosite}

\subsubsection{Limbajul C++}

Limbajul C++ este un limbaj de programare general care și este compilat în cod-mașina.
Este un limbaj multi-paradigma, cu verificare statica a tipurilor. 
Suporta programarea procedurala, orientata pe obiecte și generica.
Limbajul oferă facilitați de manipulare a memoriei la nivel scăzut.
Fiind proiectat inițial ca un limbaj pentru programarea de sisteme(sisteme integrate, kernel sisteme de operare), performața și eficienta sunt trăsături principale.

Dat fiind faptul ca este și compatibil cu limbajul C, utilizatorii C++ au la dispoziție o gama larga de librarii software din cele mai diverse ramuri de aplicații de care se pot folosii.

\subsubsection{Limbajul Python}



\subsubsection{Limbajele C++ și Python}
Pentru realizarea lucrării am ales sa folosesc C++ și Python din mai multe motive:

C++ și Python sunt doua limbaje de programare atât de diferite încât putem spune ca se afla în capete diferite ale axei limbajelor de programare.
\begin{itemize}
	\item C++ este compilat în cod-mașina, Python este interpretat
	\item Python are sistemul de tipuri dinamic și este recunoscut pentru flexibilitate
	\item C++ are sistemul de tipuri static și este recunoscut pentru eficienta
	\item Python eliberează automat memoria
\end{itemize}

Pentru multi programatori, aceste diferențe înseamna ca cele doua limbaje se complementează perfect.

\subsubsection{Librăria OpenCV}

Librăria OpenCV este cea mai populara librărie de procesare de imagini


\subsubsection{Librăria Boost}

\subsubsection{Librăria scikit-learn}

\subsubsection{Librăria Qt}



\section{Structura Lucrării}
