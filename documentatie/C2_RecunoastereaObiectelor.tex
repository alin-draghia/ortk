\chapter{Recunoașterea obiectelor}


%Recunoașterea obiectelor este o aplicate fundamentala a procesării de imagini și viziunea artificiala.
%De câteva decenii a fost, și încă este un domeniu de cercetare extensiva.
%Termenul "recunoașterea obiectelor" este folosit pentru a descrie multe aplicații și algoritmi.
%Sensul comun, de cele mai multe ori, este: date fiind cunoștințe despre înfățișarea unor obiecte, una sau mai multe imagini sunt analizate pentru a se stabili dacă exista obiectele în imagine și locația lor.
%Cu toate acestea, fiecare aplicație are cerințe și constrângeri specifice.
%Acest fapt a condus la o mare diversitate de algoritmi.
%De aceea este important ca sa avem la îndemâna biblioteci software, care sa faciliteze dezvoltarea rapida a algoritmilor de recunoaștere a obiectelor.

%Un caz special de recunoaștere a obiectelor apare foarte des, baza de date a modelelor ce trebuiesc recunoscute conține o singura clasa de obiecte, în acest caz sarcina de a detecta prezenta obiectului în imagine este simplificata.

Problema recunoașterii de obiecte se poate exprima în felul următor: Având un o baza de date cu unul sau mai multe modele de obiecte, sa se determine dacă exista obiectul în imagine și dacă exista sa se localizeze.

Unele dintre cele mai relevante lucrări din domeniu sunt: 
\begin{itemize}
	\item "Robust Real-time Object Detection" \cite{Viola01robustreal-time}
	\item "Histograms of Oriented Gradients for Human Detection" \cite{Dalal05histogramsof}
	\item "Pictorial Structures for Object Recognition" \cite{Felzenszwalb03pictorialstructures}
\end{itemize}

Dacă studiem mai atent algoritmii descriși în aceste lucrări se observa ca toate au o structura comuna și urmăresc o succesiune de operațiuni similare.
Aceste operațiuni sunt următoarele: parcurgerea imaginii în scara și spațiu, extragerea de trăsături, clasificare și post-procesarea rezultatelor.

În continuare se va discuta mai detaliat despre fiecare componenta, iar la sfârșit despre algoritmul de recunoaștere.