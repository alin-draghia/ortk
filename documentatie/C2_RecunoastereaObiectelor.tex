\chapter{Recunoașterea obiectelor}


Recunoașterea obiectelor este o aplicate fundamentala a procesării de imagini și viziunea artificiala.
De câteva decenii a fost, și încă este un domeniu de cercetare extensiva.
Termenul "recunoașterea obiectelor" este folosit pentru a descrie multe aplicații și algoritmi.
Sensul comun, de cele mai multe ori, este: date fiind cunoștințe despre înfățișarea unor obiecte, una sau mai multe imagini sunt analizate pentru a se stabili dacă exista obiectele în imagine și locația lor.
Cu toate acestea, fiecare aplicație are cerințe și constrângeri specifice.
Acest fapt a condus la o mare diversitate de algoritmi.
De aceea este important ca sa avem la îndemâna biblioteci software, care sa faciliteze dezvoltarea rapida a algoritmilor de recunoaștere a obiectelor.

Un caz special de recunoaștere a obiectelor apare foarte des, baza de date a modelelor ce trebuiesc recunoscute conține o singura clasa de obiecte, în acest caz sarcina de a detecta prezenta obiectului în imagine este simplificata.