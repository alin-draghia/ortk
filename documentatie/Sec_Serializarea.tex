\section{Serializarea}

Serializarea are un rol important în dezvoltarea algoritmilor și aplicațiilor de recunoaștere a obiectelor.
Nu este nemaiîntâlnit ca antrenarea unui astfel de algoritm sa dureze zile întregi, de aceea este întotdeauna o idee buna ca rezultatul sau chiar pașii intermediari sa fie salvați pe disc, astfel evitând pierderea datelor.
Cele mai multe limbaje de programare moderne: Java, C\#, Python ne pun la dispoziție serializare obiectelor ca trăsătura de baza a limbajului.
C++, fiind un limbaj cu un sistem de tipuri foarte complex și fără management de memorie automat, nu deține facilitați de serializare automata.

Implementarea serializării în cadrul proiectului a fost realizata folosind biblioteca Boost.Serialization.
Am ales aceasta librărie datorita suportului pentru serializarea obiectelor polimorfice, o trăsătura necesara, data fiind natura orientata pe obiect în care a fost dezvoltat proiectul.

\pagebreak